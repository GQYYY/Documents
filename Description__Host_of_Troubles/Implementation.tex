\section{Implementation}

\begin{tabular}{|c|c|c|c|}
	\hline 
	\textbf{ Intermediates} & \textbf{Source Code Website} & \textbf{Version} & \textbf{Category} \\ 
	\hline 
	\hline 
	Squid & http://www.squid-cache.org & 3.5.12 & \makecell{Transparent Cahce\\Forward Proxy\\Reverse Proxy} \\ 
	\hline 
	Apache & https://httpd.apache.org & &\makecell{Server\\ Forward Proxy\\ Reverse Proxy}\\
	\hline
\end{tabular} 

\begin{lstlisting}[title= squid/src/HttpRequest.cc line:339]
int HttpRequest::parseHeader(const char *parse_start, int len)
{
  const char *blk_start, *blk_end;
  if (!httpMsgIsolateHeaders(&parse_start, len, &blk_start, &blk_end))
  return 0;

  int result = header.parse(blk_start, blk_end);

  if (result)
  hdrCacheInit();
  return result;
}
\end{lstlisting}

\begin{lstlisting}[title=squid/src/HttpHeader.cc line:588]
int HttpHeader::parse(const char *header_start, const char *header_end)
{
  ...
  if ((e = HttpHeaderEntry::parse(field_start, field_end)) == NULL) //line 684
  ...
}
\end{lstlisting}

\begin{lstlisting}[title= squid/src/HttpHeader.cc line:1620]
/* parses and inits header entry, returns true/false */
HttpHeaderEntry * HttpHeaderEntry::
parse(const char *field_start, const char *field_end)
{
  /* note: name_start == field_start */
  const char *name_end = (const char *)memchr(field_start, ':', field_end - field_start);
  int name_len = name_end ? name_end - field_start :0;
  const char *value_start = field_start + name_len + 1; // skip ':' 
  /* note: value_end == field_end */
  ...
  /* now we know we can parse it */
  ...
  /* is it a "known" field? */
  // Header ID
  http_hdr_type id = httpHeaderIdByName(field_start, name_len, Headers, HDR_ENUM_END);
  String name;  // Header name
  String value; // Header value
  if (id < 0)
  id = HDR_OTHER;
  assert_eid(id);
  
  /* set field name */
  if (id == HDR_OTHER)
  name.limitInit(field_start, name_len);
  else
  name = Headers[id].name;
  
  /* trim field value */
  while (value_start < field_end && xisspace(*value_start))
  ++value_start;
  
  while (value_start < field_end && xisspace(field_end[-1]))
  --field_end;
  ...

  /* set field value */
  value.limitInit(value_start, field_end - value_start);
  
  debugs(55, 9, "parsed HttpHeaderEntry: '" << name << ": " << value << "'");
  
  return new HttpHeaderEntry(id, name.termedBuf(), value.termedBuf());
}
\end{lstlisting}

\begin{lstlisting}[title=squid/src/client\_side.cc line:2152]
ClientSocketContext *
parseHttpRequest(ConnStateData *csd, HttpParser *hp, HttpRequestMethod * method_p, Http::ProtocolVersion *http_ver)
{
	...
	/* Attempt to parse the first line; this'll define the method, url, version and header begin */
	r = HttpParserParseReqLine(hp);
	...
	/* Rewrite the URL in transparent or accelerator mode */
	/* NP: there are several cases to traverse here:
	*  - standard mode (forward proxy)
	*  - transparent mode (TPROXY)
	*  - transparent mode with failures
	*  - intercept mode (NAT)
	*  - intercept mode with failures
	*  - accelerator mode (reverse proxy)
	*  - internal URL
	*  - mixed combos of the above with internal URL
	*  - remote interception with PROXY protocol
	*  - remote reverse-proxy with PROXY protocol
	*/
	if (csd->transparent()) {
		/* intercept or transparent mode, properly working with no failures */
		prepareTransparentURL(csd, http, url, req_hdr);
	} else if (internalCheck(url)) {
		/* internal URL mode */
		/* prepend our name & port */
		http->uri = xstrdup(internalLocalUri(NULL, url));
		// We just re-wrote the URL. Must replace the Host: header.
		//  But have not parsed there yet!! flag for local-only handling.
		http->flags.internal = true;
	
		} else if (csd->port->flags.accelSurrogate || csd->switchedToHttps()) {
		/* accelerator mode */
		prepareAcceleratedURL(csd, http, url, req_hdr);
	}
}
\end{lstlisting}




\begin{lstlisting}[title=squid/src/client\_side.cc  line:2110 prepareTransparentURL()] 
static void
prepareTransparentURL(ConnStateData * conn, ClientHttpRequest *http, char *url, const char *req_hdr)
{
	char *host;
	char ipbuf[MAX_IPSTRLEN];
	
	if (*url != '/')
	return; /* already in good shape */
	
	/* BUG: Squid cannot deal with '*' URLs (RFC2616 5.1.2) */
	
	if ((host = mime_get_header(req_hdr, "Host")) != NULL) 
	{
		int url_sz = strlen(url) + 32 + Config.appendDomainLen +
		strlen(host);
		http->uri = (char *)xcalloc(url_sz, 1);
		snprintf(http->uri, url_sz, "%s://%s%s", AnyP::UriScheme(conn->port->transport.protocol).c_str(), host, url);
		debugs(33, 5, "TRANSPARENT HOST REWRITE: '" << http->uri <<"'");
	} 
	else 
	{
		/* Put the local socket IP address as the hostname.  */
		int url_sz = strlen(url) + 32 + Config.appendDomainLen;
		http->uri = (char *)xcalloc(url_sz, 1);
		http->getConn()->clientConnection->local.toHostStr(ipbuf,MAX_IPSTRLEN);
		snprintf(http->uri, url_sz, "%s://%s:%d%s",
		AnyP::UriScheme(http->getConn()->port->transport.protocol).c_str(),
		ipbuf, http->getConn()->clientConnection->local.port(), url);
		debugs(33, 5, "TRANSPARENT REWRITE: '" << http->uri << "'");
	}
}
\end{lstlisting}

\begin{lstlisting}[title=squid/src/mime\_header.cc  line:89]
/* returns a pointer to a field-value of the first matching field-name */
char * mime_get_header(const char *mime, const char *name)
{
	return mime_get_header_field(mime, name, NULL);
}
\end{lstlisting}


\begin{lstlisting}[title=squid/src/mime\_header.cc  line:22]
/*
* returns a pointer to a field-value of the first matching field-name where
* field-value matches prefix if any
*/
char * mime_get_header_field(const char *mime, const char *name, const char *prefix)
{
	LOCAL_ARRAY(char, header, GET_HDR_SZ);
	const char *p = NULL;
	char *q = NULL;
	char got = 0;
	const int namelen = name ? strlen(name) : 0;
	const int preflen = prefix ? strlen(prefix) : 0;
	int l;
	...
	for (p = mime; *p; p += strcspn(p, "\n\r")) {
		if (strcmp(p, "\r\n\r\n") == 0 || strcmp(p, "\n\n") == 0)
			return NULL;

		//Squid表现:If no host before: recognize. Else not recognize.
		while (xisspace(*p))  //Space-preceded Host Host前方有空格的解析 
			++p;

		if (strncasecmp(p, name, namelen))
			continue;
		if (!xisspace(p[namelen]) && p[namelen] != ':')
			continue;
		l = strcspn(p, "\n\r") + 1;
		if (l > GET_HDR_SZ)
			l = GET_HDR_SZ;

		xstrncpy(header, p, l);
		q = header;
		q += namelen;
		if (*q == ':') {
			++q;
			got = 1;
		}

		///Squid表现:If no host before: reject. Else not recognize. 有矛盾?
		while (xisspace(*q)) { //Space-succeeded Host Host后方有空格的解析 
			++q;
			got = 1;
		}

		if (got && prefix) {
			/* we could process list entries here if we had strcasestr(). */
			/* make sure we did not match a part of another field-value */
			got = !strncasecmp(q, prefix, preflen) && !xisalpha(q[preflen]);
		}

		if (got) {
			debugs(25, 5, "mime_get_header: returning '" << q << "'");
			return q;
		}
	}
	return NULL;
}
\end{lstlisting}
